\documentclass[12pt,english]{article}
\usepackage[T1]{fontenc}
\usepackage{amsmath}
\usepackage{babel}


\author{
    Meisel, Carlos \\
  \and
  Juarez, Albert\\
  \and
    Quintero, Osvaldo\\
}
\title{Task 6 - OLVHN}
\begin{document}
  \maketitle

\section{Overview}
This report contains the results and summary of the 12 step process described in Dr. Cizmas notes for computing the Operating Line. 
Also included in this report is the engine performance vartiatio with wheel speed, altitude and aircraft speed.

\section{Methodology for Computing the Operating Line}
As  we determine the Operating Line of our engine, we first define our 
operating parameters:

\begin{equation}
  \dot{m}_{air} = 1.64089 \left[\frac{kg}{s} \right]
\end{equation}

\begin{equation}
  w_{c_{n}} = 283.91025 \left[\frac{kJ}{kg}\right]
\end{equation}

\begin{equation}
  \eta_{compressor_{n}} = 0.90
\end{equation}

\begin{equation}
  \sigma_{comb} = 0.90
\end{equation}

\begin{equation}
  \eta_{turbine} = 0.94
\end{equation}

\begin{equation}
  \pi_{c_{n}} = 9.2
\end{equation}

\begin{equation}
  \lambda = 2.85377
\end{equation}

\begin{equation}
  T_{1_{n}}^{*} = 288.16 [K]
\end{equation}

\begin{equation}
  T_{3_{n}}^{*} = 1410 [K]
\end{equation}

\begin{equation}
  p_{1_{n}}^{*} =  1.01325 [bar]
\end{equation}

\begin{equation}
  p_{3_{n}}^{*} =  9.042243 [bar]
\end{equation}

\begin{equation}
  h_{3_{n}}^{*} =  1574.3477 \left[\frac{kJ}{kg}\right]
\end{equation}

\begin{equation}
  N_{n} = 52612.464822 [rpm]
\end{equation}

\begin{equation}
  \pi_{D} = 0.99
\end{equation}

\begin{equation}
  \gamma_{g} = 1.2804
\end{equation}

\begin{enumerate}
  \item Calculate the compressor work $w_{c}$, given an angular speed calculated
  in Task2, as a function of nominal compressor work and nominal angular speed.

  \begin{equation}
    w_{c} = w_{c_{n}} \left( \frac{N}{N_{n}} \right)^{x}
  \end{equation}
  Usually $x \in [1.9, 2.1]$, for convience we will start with $x = 2$.

  Note: 

  \begin{equation}
    N = 1.1N_{r}
  \end{equation}

  Recall when at nominal conditions:

  \begin{equation}
    N_{r} = \frac{N_{n}}{1.05}
  \end{equation}

  \item Estimate the compressor efficiency $\eta_{c}$, given an angular 
  speed calculated in Task2, as a function of nominal compressor efficiency and
  nominal angular speed. We start by calculated the pressure ratio $\pi_{c}^{*}$

  \begin{equation}
    \pi_{c}^{*} = \left[ \left(\pi_{c_{n}}^{\frac{\gamma-1}{\gamma}}\right)
    \frac{\eta_{c}}{\eta_{c_{n}}} \left(\frac{N}{N_{n}}\right)^{x} +1 \right]^{\frac{\gamma}{\gamma-1}}
  \end{equation}

  To begin the caluation we can start by assuming $\eta_{c} = \eta_{c_{n}}$. Once
  the pressure ratio is calculated, read that $\pi_{c}^{*}$ from the compressor map, 
  and find the corresponding $\dot{m}\frac{\sqrt{T_{1}^{*}}}{p_{1}^{*}}$. Once you have 
  that, find the corresponding $\eta_{c}$ from the compressor map. If $\eta_{c} \neq \eta_{c_{n}}$ 
  then iterate until the change is less than a allowed tolerance. For our engine,
  we allowed a tolerance of $0.01$.

  \item Calculate the $T_{3}^{*}$ from:
  
  \begin{equation}
    \pi_{c}^{*} = \frac{1+f}{\sigma_{comb}} \left( \frac{p_{3}^{*}}{\dot{m}\sqrt{T_{3}^{*}}} \right)_{n}
    \sqrt{\frac{T_{3}^{*}}{T_{1}^{*}}} \frac{\dot{m}\sqrt{T_{1}^{*}}}{p_{1}^{*}}
  \end{equation}
  Where:
  \begin{equation}
    \frac{1+f}{\sigma_{comb}} \left( \frac{p_{3}^{*}}{\dot{m}\sqrt{T_{3}^{*}}} \right)_{n} = constant
  \end{equation}
  
  From here, calculate $h_{3}^{*}$ from:

  \begin{equation}
    h_{3}^{*} = \left(\frac{1+minL}{1+\lambda minL}\right) h_{\lambda =1} + \left(\frac{(\lambda -1) minL}{1+\lambda minL}\right) h_{air}
  \end{equation}

  Where:
  \begin{itemize}
    \item $h_{\lambda}$ - enthalpy of the combustion products for $\lambda$ excess air
    \item $h_{\lambda =1}$ - enthalpy of the combustion products for stoichiometric combustion
    \item $h_{air}$ - enthalpy of the air
  \end{itemize}

  Check to see if the ratio $\frac{w_{c}}{h_{3}^{*}}$ is equal to the nominal ratio
  $\frac{w_{c_{n}}}{h_{3_{n}}^{*}}$. If not, iterate x until it is, within a reasonable tolerance of about 1\%.

  \item We are now to find the critical conditions by:
  
  \begin{equation}
    \pi_{c_{cr}}^{*} = \frac{1}{\sigma_{comb} \pi_{D}} \left[ \frac{\frac{\gamma_{g}+1}{2}}{1 - \frac{w_{c}}{h_{3}^{*}} \frac{1}{\eta_{turbine}}} \right]^{\frac{\gamma_{g}}{\gamma_{g}-1}}
  \end{equation}

  \begin{equation}
    N_{cr} = N_{n} \sqrt{\frac{\eta_{c_{n}}}{\eta_{c_{cr}}}} \frac{\pi_{c_{cr}}^{\frac{\gamma - 1}{\gamma}} -1}{\pi_{c_{n}}^{\frac{\gamma -1}{\gamma}} - 1}
  \end{equation}

  We are to then repeat steps (1)-(3) for three values of angular speed
  larger than the critical angular speed.
  
\end{enumerate}

\end{document}